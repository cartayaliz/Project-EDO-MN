\documentclass[12pt]{article}
\usepackage[utf8]{inputenc}
\usepackage[T1]{fontenc}
\usepackage[spanish]{babel}
\usepackage{amsmath, amssymb, amsthm}
\usepackage{geometry}
\geometry{margin=2.5cm}
\usepackage{graphicx}

\numberwithin{equation}{section}

\title{Trabajo pr\'actico\\
Ecuaciones Diferenciales Ordinarias\\Matemática Numérica}
\author{            Liz Cartaya Salabarría C211 \\  
                     Lianet Tamarit Tejas C211\\
                 Maya Ramón Castellanos C212                                    
                 }
\date{}
\begin{document}
\maketitle

\section*{Parte A}
1) Un reloj de agua de 12 horas se diseña con las dimensiones que se muestran en la figura 1.4.10, dada la forma de la superficie obtenida al girar la curva y f(x) alrededor del eje y. ¿Cuál debe ser esta curva, y qué radio debe tener el orificio circular del fondo para que el nivel del agua caiga a una velocidad constante de 4 pulgadas por hora in/h? 

2) Dibuje el campo de isoclinas asociado al modelo general de Torricelli:

$$ \frac{dy}{dt} =  \frac{-a}{A(y)}\sqrt{2gy}$$
e interprete el patr\´on cualitativo. Indique, por ejemplo, c\´omo var\´ıan las pendientes con \textit{y} y qu\´e papel juega A(y).
\section*{Solución:}

\section*{Modelo f\'isico y planteo de la EDO}
Partiendo de la Ley de Torricelli:
$$A(y) \frac{dy}{dt} = -a \sqrt{2gy}$$

Donde:
$A(y) = \pi x^2$  es el área de la superficie del agua a la altura $y$.
$a = \pi r^2$ es el área del orificio de salida (donde $r$ es el radio que buscamos).
$g = 32 \text{ ft/s}^2$ es la aceleración de la gravedad.$\frac{dy}{dt} = -C$ (queremos que la velocidad de bajada sea constante).

\section*{Solución exacta de la EDO:}

\section{Conversión de Unidades}

Procedemos a convertir los datos a  $\text{ft/s}$  ya que la gravedad se encuentra en $\text{ft/s}^2$
Velocidad de bajada ($C$):  $4 \text{ in/h}$.$$C = \frac{4 \text{ ft}}{12 \text{ h}} = \frac{1}{3} \text{ ft/h}$$$$C = \frac{1/3 \text{ ft}}{3600 \text{ s}} = \frac{1}{10800} \text{ ft/s}$$

\section{Determinación de la Curva $y=f(x)$}
Sustituimos $A(y)=\pi x^2$, $a=\pi r^2$ y $dy/dt = -C$ en la ecuación diferencial:$$\pi x^2 (-C) = -\pi r^2 \sqrt{2gy}$$Simplificamos $\pi$ y los signos negativos y obtenemos:$$x^2 C = r^2 \sqrt{2g} \sqrt{y} \implies x^2 = \left( \frac{r^2 \sqrt{2g}}{C} \right) \sqrt{y}$$Esto indica que $x^2$ es proporcional a $\sqrt{y}$, por lo que elevando al cuadrado, la forma de la curva es $y = k x^4$ donde k es una constante $$ k =( \frac{r^2 \sqrt{2g}}{C} ) $$
Usando las condiciones de frontera del borde superior podemos hallar la constante específica:
Cuando $x = 1 \text{ ft}$, $y = 4 \text{ ft}$.$$4 = k(1)^4 \implies k=4$$
Por tanto:
\[
\boxed{y = 4x^4}
\]

\section{Cálculo del Radio del Orificio ($r$)}
Regresamos a: 
$$\pi x^2 (-C) = -\pi r^2 \sqrt{2gy}$$
Conocemos todos los datos excepto el radio. Para obtenr el valor de r sustituimos todos los datos conocidos:
$$\pi (1)^2 \cdot \left(\frac{1}{10800}\right) = \pi r^2 \cdot \sqrt{2(32)(4)}$$
Cancelamos $\pi$ y calculamos la raíz:
$$\frac{1}{10800} = r^2 \sqrt{256}$$$$\frac{1}{10800} = r^2 (16)$$
Despejamos r:
$$r^2 = \frac{1}{16 \cdot 10800} = \frac{1}{172800} \text{ ft}^2$$


Finalmente obtenemos:
                                              $r = \sqrt{1/172800}$

Multiplicamos numerador y denominador de la fracción dentro de la raíz por 3 (un truco aritmético para obtener una raíz exacta en el denominador):$$r = \sqrt{\frac{1}{172800}} = \sqrt{\frac{1 \cdot 3}{172800 \cdot 3}} = \sqrt{\frac{3}{518400}}$$Calculamos la raíz cuadrada del denominador ($\sqrt{518400} = 720$):
\[
\boxed{r = \frac{\sqrt{3}}{720} \text{ ft}}
\]
\section*{Resumen}
Modelo: Ley de Torricelli
\[
\boxed{dV/dt = -a\sqrt{2gy}}
\]
Curva del tanque: \[
\boxed{y = 4 x^4}
\]
Radio del orificio: \[
\boxed{r = \frac{\sqrt{3}}{720} \text{ ft}}
\]

2) \section*{Patrón cualitativo:}


\section*{Variación de la pendiente}

Para \textit{y}>0  la pendiente \textit{m} es siempre negativa, ya que a,g,A(y)>0. Esto indica que la altura \textit{y} disminuye con el tiempo.

En \textit{y}=0, la pendiente es cero, lo que corresponde a un estado de equilibrio. 

Si A(\textit{y}) es constante (tanque cilíndrico), m= \alpha \textit{y}. 

Así que, a mayor \textit{y}, la pendiente es más negativa, es decir, más inclinada hacia abajo.

Si A(\textit{y}) crece con \textit{y} (tanque que se ensancha hacia arriba), el factor 1/\textit{A}(\textit{y}) reduce la magnitud de la pendiente para \textit{y} grandes, haciendo las pendientes menos inclinadas.

Si A(\textit{y}) decrece con \textit{y} (tanque que se estrecha hacia arriba), la magnitud de la pendiente aumenta para \textit{y} grandes, haciéndolas más inclinadas.

\section*{Papel de A(y)}

A(\textit{y}) modula la tasa de cambio de la altura. Un A(\textit{y}) grande reduce la altura y por tanto la pendiente, mientras que si A(\textit{y}) es pequeño la pendiente se amplifica.



\section*{Parte B}

Para estudiar cambios de r\´egimen en una variable escalar reducida
 z(t) (p.ej., una “raz\´on de vaciado adimensional” vinculada al diseño), considere la EDO con par\´ametro \mu:
 $$ \frac{dy}{dt} =  \mu + z - z^2 $$


1) Determine los puntos de equilibrio en funci\´on de \mu

2)Clasifique su estabilidad mediante $$ z ^\prime = 1 - 2z $$
 
3) Construya el diagrama de bifurcación en el plano (\mu , z)

Identifique el tipo de bifurcación
 
Comente, cualitativamente, qu\´e implicar\´ıa para el vaciado que \mu  cruce el umbral cr\´ıtico. 
 



\section*{Parte C}

Considere el siguiente sistema lineal num\´erico que idealiza la din\´amica acoplada de dos “niveles” (o un nivel y un flujo linealizado) alrededor de un punto de operaci\´on:
\[ \frac{\partial x}{\partial t} = -3x + y \]
\[ \frac{\partial y}{\partial t} = -4x - 2y \]
1) Calcule los puntos cr\´ıticos y clasif\´ıquelos (tipo y estabilidad). 

2) Verif\´ıquelo construyendo el plano de fase y explique brevemente el significado f\´ısico de las trayectorias en el contexto de niveles/flujo acoplados 

\section*{Solución:}
1) En un sistema acoplado de dos niveles tenemos dos componentes 
que interactúan entre sí tal que la evolución de cada uno depende no solo 
de sí mismo sino también del estado del otro nivel. Por ejemplo, 
dado que estamos analizando ley de Toricelli en un reloj de agua, dos tanques de agua interconectados. Las perturbaciones de un sistema afectarán al otro, y también pueden tener cambios individuales. Enfocado en el sistema con el que estamos tratando, la función puede tener dos variables donde $x(t)$ es el nivel de agua del tanque nº1
y $y(t)$, nivel de agua en el tanque nº2, y para ella el modelamiento matemático sería:

\[ \frac{\partial x}{\partial t} = a_{11}x + a_{12}y \]
\[ \frac{\partial y}{\partial t} = a_{21}x + a_{22}y \]

$a_{11}$ y $a_{22}$ serían el efecto del nivel 1 y 2 sobre sí mismos respectivamente y $a_{12}$, $a_{21}$ el del acoplamiento entre niveles.

El sistema dado es:

\[ \frac{\partial x}{\partial t} = -3x + y \]
\[ \frac{\partial y}{\partial t} = -4x - 2y \]
Para hallar sus puntos críticos, igualamos las dervidas a 0:
\[
\begin{align*}
&-3x + y = 0 \\
&y = 3x 
$\text{   ecuación 1}$ \\
&-4x - 2y = 0 
$\text{   ecuación 2}$\\
& $\text{sustituir 1 en 2}$\\
&-4x - 2(3x) = 0 \\
&-4x - 6x) = 0 \\
&-10x = 0 \\
& x = 0 \\
\end{align*}
\]
Encontrar $y$ sustituyendo $x = 0$ en 1:
\[y = 3 \cdot 0 = 0\]
Entonces el punto crítico es (0,0). Para clasificarlo, hallamos los valores propios y analizamos traza y determinante:

\[
t = -3 + (-2) = -5
\]
\[
D = 6 + 4 = 10
\]

\[
(-5)^2 < 4 \cdot 10\]
\[
25 < 40
\]


Hallar valores propios: 
\[
( -3 - \lambda )( -2 - \lambda ) + 4 = 6 + 3\lambda + 2\lambda + \lambda^2 + 4 = \lambda^2 + 5\lambda + 10
\]

\[
\lambda = \frac{-5 \pm \sqrt{25 - 4 \cdot 10}}{2}
\]

\[
= \frac{-5 \pm \sqrt{-15}}{2}
\]

\[
\lambda = \frac{-5 \pm i\sqrt{15}}{2}
\]

\[
\frac{-5}{2} < 0 \quad \text{(Parte real negativa)}
\]

Por tanto $(0,0)$ es un foco estable espiral

Estabilidad asintótica: todas las trayectorias convergen al origen,
no importa la perturbación inicial, regresa al equilibrio

2) Confirmar el resultado mediante la construcción del plano de fase.
El plano de fase es una especia de mapa con las posibles trayectorias del sistema. Los puntos críticos serán hacia donde las flechas que representan el movimiento converjan o diverjan.

Resultados observados en el plano de fase adjunto:
Las flechas giran en espiral hacia el centro y las trayectorias terminan en (0,0), el origen, confirmando la convergencia y estabilidad.

La interpretación física es que los cambios en los niveles producen oscilaciones amortiguadas donde el sistema se autocorrige y se termina estabilizando. Ocurre un intercambio donde el exceso en un elemento se transfiere al otro, redistribuyéndose. Por tanto, tras cualquier perturbación en los niveles, estos eventualmente regresan al equilibrio.


\end{document}