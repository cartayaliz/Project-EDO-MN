\documentclass[12pt]{article}
\usepackage[utf8]{inputenc}
\usepackage[T1]{fontenc}
\usepackage[spanish]{babel}
\usepackage{amsmath, amssymb, amsthm}
\usepackage{geometry}
\geometry{margin=2.5cm}
\usepackage{graphicx}

% Opcional: Para mejorar el formato de los autores si son muchos
\usepackage{authblk}

\numberwithin{equation}{section}

\title{Trabajo práctico\\
Ecuaciones Diferenciales Ordinarias\\Matemática Numérica}
\author{Liz Cartaya Salabarría C211 \\ Lianet Tamarit Tejas C211\\Maya Ramón Castellanos C212}
\date{}

\begin{document}
\maketitle

\section*{Parte A}

1) Un reloj de agua de 12 horas se diseña con las dimensiones que se muestran en la figura 1.4.10, dada la forma de la superficie obtenida al girar la curva $y=f(x)$ alrededor del eje $y$. ¿Cuál debe ser esta curva, y qué radio debe tener el orificio circular del fondo para que el nivel del agua caiga a una velocidad constante de 4 pulgadas por hora (in/h)? 

2) Dibuje el campo de isoclinas asociado al modelo general de Torricelli:

\[ \frac{dy}{dt} =  \frac{-a}{A(y)}\sqrt{2gy} \]

e interprete el patrón cualitativo. Indique, por ejemplo, cómo varían las pendientes con $y$ y qué papel juega $A(y)$.

\section*{Solución:}

\section*{Modelo físico y planteo de la EDO}
Partiendo de la Ley de Torricelli:
\[ A(y) \frac{dy}{dt} = -a \sqrt{2gy} \]

Donde:
\begin{itemize}
\item $A(y) = \pi x^2$ es el área de la superficie del agua a la altura $y$. \item $a = \pi r^2$ es el área del orificio de salida (donde $r$ es el radio que buscamos). \item $g = 32 \text{ ft/s}^2$ es la aceleración de la gravedad.
\item $\frac{dy}{dt} = -C$ (queremos que la velocidad de bajada sea constante).
\end{itemize}

\section{Conversión de Unidades}
Procedemos a convertir los datos a $\text{ft/s}$ ya que la gravedad se encuentra en $\text{ft/s}^2$.

Velocidad de bajada ($C$): $4 \text{ in/h}$.
\[ C = \frac{4 \text{ ft}}{12 \text{ h}} = \frac{1}{3} \text{ ft/h} \]
\[ C = \frac{1/3 \text{ ft}}{3600 \text{ s}} = \frac{1}{10800} \text{ ft/s} \]

\section{Determinación de la Curva $y=f(x)$}
Sustituimos $A(y)=\pi x^2$, $a=\pi r^2$ y $dy/dt = -C$ en la ecuación diferencial:
\[ \pi x^2 (-C) = -\pi r^2 \sqrt{2gy} \]

Simplificamos $\pi$ y los signos negativos y obtenemos:
\[ x^2 C = r^2 \sqrt{2g} \sqrt{y} \implies x^2 = \left( \frac{r^2 \sqrt{2g}}{C} \right) \sqrt{y} \]

Esto indica que $x^2$ es proporcional a $\sqrt{y}$, por lo que elevando al cuadrado, la forma de la curva es $y = k x^4$ donde $k$ es una constante:
\[ k = \left( \frac{r^2 \sqrt{2g}}{C} \right) \]

Usando las condiciones de frontera del borde superior podemos hallar la constante específica:
Cuando $x = 1 \text{ ft}$, $y = 4 \text{ ft}$.
\[ 4 = k(1)^4 \implies k=4 \]
Por tanto:
\[ \boxed{y = 4x^4} \]

\section{Cálculo del Radio del Orificio ($r$)}
Regresamos a: 
\[ \pi x^2 (-C) = -\pi r^2 \sqrt{2gy} \]

Conocemos todos los datos excepto el radio. Para obtener el valor de $r$ sustituimos todos los datos conocidos:
\[ \pi (1)^2 \cdot \left(\frac{1}{10800}\right) = \pi r^2 \cdot \sqrt{2(32)(4)} \]

Cancelamos $\pi$ y calculamos la raíz:
\[ \frac{1}{10800} = r^2 \sqrt{256} \]
\[ \frac{1}{10800} = r^2 (16) \]

Despejamos $r$:
\[ r^2 = \frac{1}{16 \cdot 10800} = \frac{1}{172800} \text{ ft}^2 \]

Finalmente obtenemos:
\[ r = \sqrt{1/172800} \]

Multiplicamos numerador y denominador de la fracción dentro de la raíz por 3 (un truco aritmético para obtener una raíz exacta en el denominador):
\[ r = \sqrt{\frac{1}{172800}} = \sqrt{\frac{1 \cdot 3}{172800 \cdot 3}} = \sqrt{\frac{3}{518400}} \]

Calculamos la raíz cuadrada del denominador ($\sqrt{518400} = 720$):
\[ \boxed{r = \frac{\sqrt{3}}{720} \text{ ft}} \]

\section*{Resumen}
Modelo: Ley de Torricelli
\[ \boxed{\frac{dV}{dt} = -a\sqrt{2gy}} \]
Curva del tanque: 
\[ \boxed{y = 4 x^4} \]
Radio del orificio: 
\[ \boxed{r = \frac{\sqrt{3}}{720} \text{ ft}} \]

\section*{Parte A.2: Patrón cualitativo}

\section*{Variación de la pendiente}
Para $y>0$ la pendiente $m$ es siempre negativa, ya que $a,g,A(y)>0$. Esto indica que la altura $y$ disminuye con el tiempo.

En $y=0$, la pendiente es cero, lo que corresponde a un estado de equilibrio. 

Si $A(y)$ es constante (tanque cilíndrico), $m \propto \sqrt{y}$. Así que, a mayor $y$, la pendiente es más negativa, es decir, más inclinada hacia abajo.

Si $A(y)$ crece con $y$ (tanque que se ensancha hacia arriba), el factor $1/A(y)$ reduce la magnitud de la pendiente para $y$ grandes, haciendo las pendientes menos inclinadas.

Si $A(y)$ decrece con $y$ (tanque que se estrecha hacia arriba), la magnitud de la pendiente aumenta para $y$ grandes, haciéndolas más inclinadas.

\section*{Papel de $A(y)$}
$A(y)$ modula la tasa de cambio de la altura. Un $A(y)$ grande reduce la velocidad de cambio de la altura y por tanto la pendiente, mientras que si $A(y)$ es pequeño la pendiente se amplifica.

\section*{Parte B}

Para estudiar cambios de régimen en una variable escalar reducida $z(t)$ (p.ej., una "razón de vaciado adimensional" vinculada al diseño), considere la EDO con parámetro $\mu$:
\[ \frac{dz}{dt} = \mu + z - z^2 \]

1) Determine los puntos de equilibrio en función de $\mu$.

2) Clasifique su estabilidad mediante:
\[ z' = 1 - 2z \]

3) Construya el diagrama de bifurcación en el plano $(\mu , z)$.
Identifique el tipo de bifurcación. Comente, cualitativamente, qué implicaría para el vaciado que $\mu$ cruce el umbral crítico.
\section*{1. Determinación de los Puntos de Equilibrio}

Los puntos de equilibrio, $z^*$, se obtienen igualando la velocidad de cambio a cero, $\frac{dz}{dt} = 0$:
$$ \mu + z - z^2 = 0 $$
Reordenando a la forma cuadrática $az^2 + bz + c = 0$:
$$ z^2 - z - \mu = 0 $$
Aplicando la fórmula general para la solución de ecuaciones cuadráticas:
$$ z^* = \frac{-(-1) \pm \sqrt{(-1)^2 - 4(1)(-\mu)}}{2(1)} $$
Los puntos de equilibrio en función de $\mu$ son:
$$ z_{1,2}^*(\mu) = \frac{1 \pm \sqrt{1 + 4\mu}}{2} $$
Para que existan puntos de equilibrio reales, el discriminante debe ser no negativo: $1 + 4\mu \geq 0$, lo que implica que la bifurcación ocurre en $\mu = -\frac{1}{4}$.

\section*{2. Clasificación de la Estabilidad}



\subsection*{Análisis de $z_1^*$ (Rama Inferior)}
$$ z_1^* = \frac{1 - \sqrt{1 + 4\mu}}{2} $$
Evaluando la derivada:
$$ f'(z_1^*) = 1 - 2\left(\frac{1 - \sqrt{1 + 4\mu}}{2}\right) = 1 - (1 - \sqrt{1 + 4\mu}) = \sqrt{1 + 4\mu} $$
Para $\mu > -\frac{1}{4}$, $\sqrt{1 + 4\mu} > 0$. Por lo tanto, $z_1^*$ es \textbf{inestable} (repulsor).

\subsection*{Análisis de $z_2^*$ (Rama Superior)}
$$ z_2^* = \frac{1 + \sqrt{1 + 4\mu}}{2} $$
Evaluando la derivada:
$$ f'(z_2^*) = 1 - 2\left(\frac{1 + \sqrt{1 + 4\mu}}{2}\right) = 1 - (1 + \sqrt{1 + 4\mu}) = -\sqrt{1 + 4\mu} $$
Para $\mu > -\frac{1}{4}$, $-\sqrt{1 + 4\mu} < 0$. Por lo tanto, $z_2^*$ es \textbf{asintóticamente estable} (atractor).

\section*{3. Diagrama de Bifurcación y Tipo}

\begin{figure}[h!]
    \centering
    \includegraphics[width=0.7\textwidth]{grafico parte b.png}
    \caption{Diagrama de Bifurcación. Las líneas continuas representan puntos estables ($z_2^*$) y las discontinuas, puntos inestables ($z_1^*$). La bifurcación ocurre en $\mu_c = -0.25$.}
    \label{fig:bifurcacion_silla_nodo}
\end{figure}


\subsection*{Tipo de Bifurcación}

 Según la forma de aparición/desaparición de equilibrios:

-Dos equilibrios (uno estable y otro inestable) nacen o se destruyen en un punto.

-Ocurre cuando el discriminante se anula y f, junto con su derivada son cero al mismo tiempo.

Esa es la definición típica de una bifurcación nodo silla.

\subsection*{Implicación Cualitativa para el Vaciado}

La variable $z(t)$ es la razón de vaciado adimensional y $\mu$ es un parámetro de diseño.

\begin{itemize}
 \item Si $\mu < -\dfrac{1}{4}$: el tanque se vacía completamente (no hay equilibrio).
\item Si $\mu = -\dfrac{1}{4}$: hay un cambio de régimen (\textbf{bifurcación}).
 \item Si $\mu > -\dfrac{1}{4}$: puede alcanzarse un \textbf{nivel estable de líquido} (equilibrio).
\end{itemize}



\section*{Parte C}

Considere el siguiente sistema lineal numérico que idealiza la dinámica acoplada de dos "niveles" (o un nivel y un flujo linealizado) alrededor de un punto de operación:
\[ \frac{dx}{dt} = -3x + y \]
\[ \frac{dy}{dt} = -4x - 2y \]

1) Calcule los puntos críticos y clasifíquelos (tipo y estabilidad). 

2) Verifíquelo construyendo el plano de fase y explique brevemente el significado físico de las trayectorias en el contexto de niveles/flujo acoplados.

\section*{Solución Parte C:}

1) En un sistema acoplado de dos niveles tenemos dos componentes que interactúan entre sí tal que la evolución de cada uno depende no solo de sí mismo sino también del estado del otro nivel. El sistema dado es:

\[ \frac{dx}{dt} = -3x + y \]
\[ \frac{dy}{dt} = -4x - 2y \]

Para hallar sus puntos críticos, igualamos las derivadas a 0:
\begin{align*}
-3x + y &= 0 && \text{(ecuación 1)} \\
y &= 3x \\
-4x - 2y &= 0 && \text{(ecuación 2)} \\
\text{Sustituimos 1 en 2:} & \\
-4x - 2(3x) &= 0 \\
-4x - 6x &= 0 \\
-10x &= 0 \\
x &= 0 
\end{align*}

Encontrar $y$ sustituyendo $x = 0$ en la ecuación 1:
\[ y = 3 \cdot 0 = 0 \]

Entonces el punto crítico es $(0,0)$. Para clasificarlo, hallamos los valores propios y analizamos traza y determinante:
\[ \text{Traza } \tau = -3 + (-2) = -5 \]
\[ \text{Det } \Delta = (-3)(-2) - (1)(-4) = 6 + 4 = 10 \]

Discriminante:
\[ \tau^2 - 4\Delta = (-5)^2 - 4(10) = 25 - 40 = -15 \]

Como $\tau^2 < 4\Delta$, los valores propios son complejos.
Hallar valores propios: 
\[ \lambda^2 - \tau\lambda + \Delta = 0 \]
\[ \lambda^2 + 5\lambda + 10 = 0 \]
\[ \lambda = \frac{-5 \pm \sqrt{25 - 40}}{2} = \frac{-5 \pm \sqrt{-15}}{2} \]
\[ \lambda = \frac{-5}{2} \pm i\frac{\sqrt{15}}{2} \]

Como la parte real es negativa ($\text{Re}(\lambda) = -2.5 < 0$), el punto $(0,0)$ es un **foco estable (o espiral sumidero)**.

**Estabilidad asintótica:** todas las trayectorias convergen al origen; no importa la perturbación inicial, el sistema regresa al equilibrio.

2) Confirmar el resultado mediante la construcción del plano de fase.
El plano de fase es una especie de mapa con las posibles trayectorias del sistema. Los puntos críticos serán hacia donde las flechas que representan el movimiento converjan o diverjan.

**Resultados observados en el plano de fase:**
Las flechas giran en espiral hacia el centro y las trayectorias terminan en $(0,0)$, el origen, confirmando la convergencia y estabilidad.

La interpretación física es que los cambios en los niveles producen oscilaciones amortiguadas donde el sistema se autocorrige y se termina estabilizando. Ocurre un intercambio donde el exceso en un elemento se transfiere al otro, redistribuyéndose. Por tanto, tras cualquier perturbación en los niveles, estos eventualmente regresan al equilibrio.

\end{document}